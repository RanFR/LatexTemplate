\documentclass{ranarticle}

\title{ranarticle.cls模板使用示例}
\author{RanFR}
\date{\today}

\begin{document}

\maketitle

\begin{abstract}
    在这个示例教程中,我们将演示如何使用ranarticle.cls模板来创建一个简单的中文文档。这个文档包括数学公式、表格、图片、列表、代码和参考文献等内容。
\end{abstract}

\section{数学公式示例}

这是几个数学公式示例:

\begin{equation}
    E = mc^2
\end{equation}

\begin{equation}
    \int_0^1 x^2 \, dx = \frac{1}{3}
\end{equation}

\section{表格示例}

\subsection{简单表格示例}

\subsubsection{基本表格}

这是一个简单表格示例,使用tabular包,见表\ref{tab:basic_tabular_table}。

\begin{table}[htbp]
    \centering
    \caption{简单表格示例}
    \label{tab:basic_tabular_table}
    \begin{tabular}{|c|c|c|}
        \hline
        项目 & 数量 & 单价 \\
        \hline
        A    & 5    & 20   \\
        B    & 3    & 25   \\
        C    & 8    & 15   \\
        \hline
    \end{tabular}
\end{table}

\subsubsection{自适应基本表格}

这是一个自适应基本表格示例,使用tabularx包,见表\ref{tab:basic_tabularx_table}。

\begin{table}[htbp]
    \centering
    \caption{一个表格}
    \label{tab:basic_tabularx_table}
    \begin{tabularx}{\textwidth}{
            | >{\raggedright\arraybackslash}X
            | >{\centering\arraybackslash}X
            | >{\raggedleft\arraybackslash}X
            |}
        \hline
        项目(左) & 数量(中) & 单价(右) \\
        \hline
        A          & 5          & 20         \\
        \hline
        B          & 3          & 25         \\
        \hline
    \end{tabularx}
\end{table}

\subsection{多行合并表格示例}

这是一个多行表格示例。

\begin{table}[htbp]
    \centering
    \caption{多行合并表格示例}
    \begin{tabular}{|c|c|c|}
        \hline
        项目                    & 数值1 & 数值2 \\
        \hline
        \multirow{2}{*}{合并行} & 10    & 20    \\
        \cline{2-3}
                                & 30    & 40    \\
        \hline
        \makecell{多行内容                      \\第二行} & 50 & 60 \\
        \hline
    \end{tabular}
\end{table}

\subsection{多列合并表格示例}

这是一个多列合并表格示例。

\begin{table}[htbp]
    \centering
    \caption{多列合并表格示例}
    \begin{tabular}{|c|c|c|c|}
        \hline
        \multicolumn{2}{|c|}{项目} & 数值1 & 数值2      \\
        \hline
        \multirow{2}{*}{项目a}     & 属性a & 10    & 20 \\
        \cline{2-4}
                                   & 属性b & 30    & 40 \\
        \hline
    \end{tabular}
\end{table}

\section{列表示例}
\subsection{有序列表}
这是一个有序列表示例:
\begin{enumerate}
    \item 列表项 1
    \item 列表项 2
    \item 列表项 3
          \begin{enumerate}
              \item 子列表项 1
              \item 子列表项 2
          \end{enumerate}
\end{enumerate}

\subsection{无序列表}

这是一个无序列表示例:

\begin{itemize}
    \item 列表项 1
    \item 列表项 2
    \item 列表项 3
          \begin{itemize}
              \item 子列表项 1
              \item 子列表项 2
          \end{itemize}
\end{itemize}

\section{图片示例}

这是一张图片示例:

\begin{figure}[htbp]
    \centering
    \includegraphics[width=0.6\textwidth]{assets/example.png}
    \caption{一张图片}
\end{figure}

\section{代码示例}

这是一个Python代码示例:

\begin{minted}{python}
    def hello_world():
        print("Hello, World!")

    hello_world()
\end{minted}

\section{伪代码示例}

这是一个简单的算法伪代码示例:

\begin{algorithm}[H]
    \caption{计算斐波那契数列的第n项}
    \KwIn{正整数$n$}
    \KwOut{第$n$项斐波那契数列的值}
    \SetKwProg{Fn}{Function}{\string:}{end}
    \Fn{Fibonacci($n$)}{
        \If{$n=1$ or $n=2$}{
            \Return 1;
        }
        $F_1\leftarrow 1$; $F_2\leftarrow 1$;
        \For{$i\leftarrow 3$ \KwTo $n$}{
            $F_i\leftarrow F_{i-1} + F_{i-2}$;
        }
        \Return $F_n$;
    }
\end{algorithm}

\section{符号示例}

由\text{gensymb}包提供的符号示例:

温度:25\degree C

欧姆:53 \ohm

\end{document}
